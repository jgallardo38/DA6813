% Options for packages loaded elsewhere
\PassOptionsToPackage{unicode}{hyperref}
\PassOptionsToPackage{hyphens}{url}
%
\documentclass[
]{article}
\usepackage{amsmath,amssymb}
\usepackage{iftex}
\ifPDFTeX
  \usepackage[T1]{fontenc}
  \usepackage[utf8]{inputenc}
  \usepackage{textcomp} % provide euro and other symbols
\else % if luatex or xetex
  \usepackage{unicode-math} % this also loads fontspec
  \defaultfontfeatures{Scale=MatchLowercase}
  \defaultfontfeatures[\rmfamily]{Ligatures=TeX,Scale=1}
\fi
\usepackage{lmodern}
\ifPDFTeX\else
  % xetex/luatex font selection
\fi
% Use upquote if available, for straight quotes in verbatim environments
\IfFileExists{upquote.sty}{\usepackage{upquote}}{}
\IfFileExists{microtype.sty}{% use microtype if available
  \usepackage[]{microtype}
  \UseMicrotypeSet[protrusion]{basicmath} % disable protrusion for tt fonts
}{}
\makeatletter
\@ifundefined{KOMAClassName}{% if non-KOMA class
  \IfFileExists{parskip.sty}{%
    \usepackage{parskip}
  }{% else
    \setlength{\parindent}{0pt}
    \setlength{\parskip}{6pt plus 2pt minus 1pt}}
}{% if KOMA class
  \KOMAoptions{parskip=half}}
\makeatother
\usepackage{xcolor}
\usepackage[margin=1in]{geometry}
\usepackage{longtable,booktabs,array}
\usepackage{calc} % for calculating minipage widths
% Correct order of tables after \paragraph or \subparagraph
\usepackage{etoolbox}
\makeatletter
\patchcmd\longtable{\par}{\if@noskipsec\mbox{}\fi\par}{}{}
\makeatother
% Allow footnotes in longtable head/foot
\IfFileExists{footnotehyper.sty}{\usepackage{footnotehyper}}{\usepackage{footnote}}
\makesavenoteenv{longtable}
\usepackage{graphicx}
\makeatletter
\def\maxwidth{\ifdim\Gin@nat@width>\linewidth\linewidth\else\Gin@nat@width\fi}
\def\maxheight{\ifdim\Gin@nat@height>\textheight\textheight\else\Gin@nat@height\fi}
\makeatother
% Scale images if necessary, so that they will not overflow the page
% margins by default, and it is still possible to overwrite the defaults
% using explicit options in \includegraphics[width, height, ...]{}
\setkeys{Gin}{width=\maxwidth,height=\maxheight,keepaspectratio}
% Set default figure placement to htbp
\makeatletter
\def\fps@figure{htbp}
\makeatother
\setlength{\emergencystretch}{3em} % prevent overfull lines
\providecommand{\tightlist}{%
  \setlength{\itemsep}{0pt}\setlength{\parskip}{0pt}}
\setcounter{secnumdepth}{-\maxdimen} % remove section numbering
\ifLuaTeX
  \usepackage{selnolig}  % disable illegal ligatures
\fi
\IfFileExists{bookmark.sty}{\usepackage{bookmark}}{\usepackage{hyperref}}
\IfFileExists{xurl.sty}{\usepackage{xurl}}{} % add URL line breaks if available
\urlstyle{same}
\hypersetup{
  pdftitle={Bookbinder Study Case},
  pdfauthor={Alex Martinez, Josh Gardner, Cameron Playle, and Guillermo Gallardo},
  hidelinks,
  pdfcreator={LaTeX via pandoc}}

\title{Bookbinder Study Case}
\author{Alex Martinez, Josh Gardner, Cameron Playle, and Guillermo
Gallardo}
\date{2024-09-25}

\begin{document}
\maketitle

\hypertarget{paper-starts-here}{%
\subparagraph{\texorpdfstring{\textbf{Paper Starts
Here}}{Paper Starts Here}}\label{paper-starts-here}}

\hypertarget{executive-summary}{%
\section{Executive Summary}\label{executive-summary}}

\emph{Brief introduction of problem. Summarizes key findings. Summarizes
insights behind key findings.}

\hypertarget{our-problem}{%
\section{Our Problem}\label{our-problem}}

\emph{Clear description of the problem, from an application and
theoretical point of view. Outlines the report.}

For this study case our goal is to evaluate how effective three
different models are and compare them with the option of creating this
campaign without a model. We are trying to determine which model will
provide the best balance between cost savings and profit. By analyzing
the performance of each model and comparing ti against the campaign, we
will identify the most cost-effective approach that maximizes profit.

The three models we are comparing are the Linear Model (LM), Generalized
Linear Model (GLM), and Support Vector Machine (SVM). Although we
anticipate that the linear model may not perform well, we are still
interested in understanding why it may not be the best fit for this case
study. This exploration will help us gain valuable insights into the
limitations of the linear model in this context and guide our
decision-making process.

\textbf{He said that we don't have to use the 50k population for our
calculation to find which way would be best for the campaign. Email
everyone in the 2300 or just a group of people based on our model.}

\hypertarget{literature-review}{%
\section{Literature Review}\label{literature-review}}

\emph{Discusses and cites existing works in the theoretical and
application realm.}

\hypertarget{methods}{%
\section{Methods}\label{methods}}

\emph{Discusses types of variables, sample size, and sampling techniques
(if any). Discusses the model(s) and its assumptions and limitations.}

Our dataset was given to us divided into training and test sets. The
training set includes 1,600 observations, while the test set contains
2,300 observations. The dataset consists of 12 variables, with one
variable (observation) being removed. Two variables were converted into
factors: gender and choice, with choice serving as our response
variable. The remaining variables are numerical. For some of these
numerical variables, including Last\_purchased, we will transform them
into categorical variables to see how it would impact our modeling
process.

Add stuff about unabalanced dataset?

GLM:

SVM: balance vs unbalanced comparison?

LDA maybe?

\hypertarget{data}{%
\section{Data}\label{data}}

\emph{Discusses how data was handled, i.e.~cleaned and preprocessed.
Discusses distributions, correlations, etc.}

\hypertarget{clean-up}{%
\subsection{Clean up}\label{clean-up}}

We modified two variables into factors, and we created categories for
ADD STUFF HERE?

\hypertarget{correlation}{%
\subsection{Correlation}\label{correlation}}

The graph below highlights the variables with the highest correlations.
We observed that \emph{first\_purchased} and \emph{last\_purchased}
exhibit the strongest correlation. Based on this, we decided to create
labels for these two variables to test their potential impact on our
model's performance.. \textbf{ADD STUFF ABOUT RESULTS. DID IT WORK OR
DID IT MAKE IT WORST?}

\includegraphics{BBBC-Case-Study_files/figure-latex/Correlation Plot-1.pdf}

\hypertarget{results}{%
\section{Results}\label{results}}

\emph{Presents and discusses the results from model(s). Discusses
relationships between covariates and response, if possible, and provides
deep insights behind relationships in the context of the application.}

\hypertarget{glm-results}{%
\subsection{GLM Results}\label{glm-results}}

Here we will be discussing the outputs from our logistic model we
applied to the data set to best predict if someone will purchase a book
- Choice (1/0). We first begin with all the variables in the data set to
see which independent variables are significant.

\begin{verbatim}
## 
## Call:
## glm(formula = Choice ~ . - Observation, data = bbc_train)
## 
## Coefficients:
##                    Estimate Std. Error t value Pr(>|t|)    
## (Intercept)       0.3642284  0.0307411  11.848  < 2e-16 ***
## Gender           -0.1309205  0.0200303  -6.536 8.48e-11 ***
## Amount_purchased  0.0002736  0.0001110   2.464   0.0138 *  
## Frequency        -0.0090868  0.0021791  -4.170 3.21e-05 ***
## Last_purchase     0.0970286  0.0135589   7.156 1.26e-12 ***
## First_purchase   -0.0020024  0.0018160  -1.103   0.2704    
## P_Child          -0.1262584  0.0164011  -7.698 2.41e-14 ***
## P_Youth          -0.0963563  0.0201097  -4.792 1.81e-06 ***
## P_Cook           -0.1414907  0.0166064  -8.520  < 2e-16 ***
## P_DIY            -0.1352313  0.0197873  -6.834 1.17e-11 ***
## P_Art             0.1178494  0.0194427   6.061 1.68e-09 ***
## ---
## Signif. codes:  0 '***' 0.001 '**' 0.01 '*' 0.05 '.' 0.1 ' ' 1
## 
## (Dispersion parameter for gaussian family taken to be 0.1434751)
## 
##     Null deviance: 300.00  on 1599  degrees of freedom
## Residual deviance: 227.98  on 1589  degrees of freedom
## AIC: 1447
## 
## Number of Fisher Scoring iterations: 2
\end{verbatim}

With all the dependent variables, last\_Purchase has the largest VIF so
we decide to remove that from our model.

\begin{verbatim}
## 
## Call:
## glm(formula = Choice ~ . - Observation - Last_purchase, data = bbc_train)
## 
## Coefficients:
##                    Estimate Std. Error t value Pr(>|t|)    
## (Intercept)       0.3926595  0.0309609  12.682  < 2e-16 ***
## Gender           -0.1290720  0.0203424  -6.345 2.89e-10 ***
## Amount_purchased  0.0003518  0.0001122   3.135 0.001753 ** 
## Frequency        -0.0157943  0.0019980  -7.905 4.97e-15 ***
## First_purchase    0.0046036  0.0015884   2.898 0.003803 ** 
## P_Child          -0.0502183  0.0126891  -3.958 7.90e-05 ***
## P_Youth          -0.0225339  0.0175326  -1.285 0.198888    
## P_Cook           -0.0667467  0.0131127  -5.090 4.00e-07 ***
## P_DIY            -0.0606486  0.0170835  -3.550 0.000396 ***
## P_Art             0.1916012  0.0167447  11.443  < 2e-16 ***
## ---
## Signif. codes:  0 '***' 0.001 '**' 0.01 '*' 0.05 '.' 0.1 ' ' 1
## 
## (Dispersion parameter for gaussian family taken to be 0.1480058)
## 
##     Null deviance: 300.00  on 1599  degrees of freedom
## Residual deviance: 235.33  on 1590  degrees of freedom
## AIC: 1495.8
## 
## Number of Fisher Scoring iterations: 2
\end{verbatim}

\begin{verbatim}
##           Gender Amount_purchased        Frequency   First_purchase 
##         1.005634         1.235982         2.651820         7.182666 
##          P_Child          P_Youth           P_Cook            P_DIY 
##         1.949849         1.307915         2.009609         1.457362 
##            P_Art 
##         1.634878
\end{verbatim}

First\_purchase also has a large VIF so we will remove that from our
model.

\begin{verbatim}
## 
## Call:
## glm(formula = Choice ~ . - Observation - Last_purchase - First_purchase, 
##     data = bbc_train)
## 
## Coefficients:
##                    Estimate Std. Error t value Pr(>|t|)    
## (Intercept)       0.3731865  0.0302933  12.319  < 2e-16 ***
## Gender           -0.1263728  0.0203683  -6.204 6.99e-10 ***
## Amount_purchased  0.0003688  0.0001123   3.283  0.00105 ** 
## Frequency        -0.0112345  0.0012344  -9.101  < 2e-16 ***
## P_Child          -0.0275983  0.0100284  -2.752  0.00599 ** 
## P_Youth          -0.0014841  0.0159946  -0.093  0.92609    
## P_Cook           -0.0428346  0.0102155  -4.193 2.90e-05 ***
## P_DIY            -0.0384262  0.0153017  -2.511  0.01213 *  
## P_Art             0.2183323  0.0140081  15.586  < 2e-16 ***
## ---
## Signif. codes:  0 '***' 0.001 '**' 0.01 '*' 0.05 '.' 0.1 ' ' 1
## 
## (Dispersion parameter for gaussian family taken to be 0.1486942)
## 
##     Null deviance: 300.00  on 1599  degrees of freedom
## Residual deviance: 236.57  on 1591  degrees of freedom
## AIC: 1502.2
## 
## Number of Fisher Scoring iterations: 2
\end{verbatim}

\begin{verbatim}
##           Gender Amount_purchased        Frequency          P_Child 
##         1.003526         1.232595         1.007587         1.212223 
##          P_Youth           P_Cook            P_DIY            P_Art 
##         1.083475         1.214043         1.163794         1.138879
\end{verbatim}

Finally, P\_Youth has a p-value \textgreater{} 0.05 so we will remove
that to have our final model

\begin{verbatim}
## 
## Call:
## glm(formula = Choice ~ . - Observation - Last_purchase - First_purchase - 
##     P_Youth, data = bbc_train)
## 
## Coefficients:
##                    Estimate Std. Error t value Pr(>|t|)    
## (Intercept)       0.3730697  0.0302577  12.330  < 2e-16 ***
## Gender           -0.1263681  0.0203619  -6.206 6.91e-10 ***
## Amount_purchased  0.0003679  0.0001118   3.289  0.00103 ** 
## Frequency        -0.0112341  0.0012341  -9.103  < 2e-16 ***
## P_Child          -0.0276675  0.0099975  -2.767  0.00572 ** 
## P_Cook           -0.0429132  0.0101772  -4.217 2.62e-05 ***
## P_DIY            -0.0385786  0.0152086  -2.537  0.01129 *  
## P_Art             0.2182592  0.0139816  15.610  < 2e-16 ***
## ---
## Signif. codes:  0 '***' 0.001 '**' 0.01 '*' 0.05 '.' 0.1 ' ' 1
## 
## (Dispersion parameter for gaussian family taken to be 0.1486016)
## 
##     Null deviance: 300.00  on 1599  degrees of freedom
## Residual deviance: 236.57  on 1592  degrees of freedom
## AIC: 1500.2
## 
## Number of Fisher Scoring iterations: 2
\end{verbatim}

\begin{verbatim}
##           Gender Amount_purchased        Frequency          P_Child 
##         1.003520         1.222450         1.007578         1.205526 
##           P_Cook            P_DIY            P_Art 
##         1.205706         1.150392         1.135282
\end{verbatim}

Before we get into the confusion matrix. Lets explain the relationship
of each independent variable to the dependent variable. We first compute
the exponential of our coefficient ratio to get the odds ratio.

\begin{longtable}[]{@{}llr@{}}
\caption{Odds Ratios from the Logistic Model}\tabularnewline
\toprule\noalign{}
& Variable & Odds Ratio \\
\midrule\noalign{}
\endfirsthead
\toprule\noalign{}
& Variable & Odds Ratio \\
\midrule\noalign{}
\endhead
\bottomrule\noalign{}
\endlastfoot
Gender & Gender (Male) & 0.8812904 \\
Amount\_purchased & Amount Purchased & 1.0003680 \\
Frequency & Purchase Frequency & 0.9888287 \\
P\_Child & Child Books Purchased & 0.9727118 \\
P\_Cook & Cook Books Purchased & 0.9579945 \\
P\_DIY & DIY Books Purchased & 0.9621561 \\
P\_Art & Art Books Purchased & 1.2439095 \\
\end{longtable}

We observe the following key findings from the model:

\begin{itemize}
\item
  Gender (Male): Males decrease the odds of a client buying a book by a
  factor of 0.88.
\item
  Amount of Books Purchased: A larger amount of books purchased
  increases the odds of a client buying a book by a factor of 1.
\item
  Purchase Frequency: A higher frequency of books purchased decreases
  the odds of a client buying a book by a factor of 0.99.
\item
  Child Books Purchased: A higher purchase of child books increases the
  odds of a client buying a book by a factor of 0.97.
\item
  DIY Books Purchased: A higher purchase of DIY books increases the odds
  of a client buying a book by a factor of 0.96.
\item
  Art Books Purchased: A higher purchase of art books increases the odds
  of a client buying a book by a factor of 1.24.
\end{itemize}

With this model, we then use it on our bbc\_test sample to see how well
it predicts and determine the most optimal cutoff to have the highest
Sensitivity. Here are the following results:

\begin{verbatim}
## Confusion Matrix and Statistics
## 
##           Reference
## Prediction    0    1
##          0 2005   91
##          1  140   64
##                                           
##                Accuracy : 0.8996          
##                  95% CI : (0.8866, 0.9116)
##     No Information Rate : 0.9326          
##     P-Value [Acc > NIR] : 1.000000        
##                                           
##                   Kappa : 0.3032          
##                                           
##  Mcnemar's Test P-Value : 0.001588        
##                                           
##             Sensitivity : 0.41290         
##             Specificity : 0.93473         
##          Pos Pred Value : 0.31373         
##          Neg Pred Value : 0.95658         
##              Prevalence : 0.06739         
##          Detection Rate : 0.02783         
##    Detection Prevalence : 0.08870         
##       Balanced Accuracy : 0.67382         
##                                           
##        'Positive' Class : 1               
## 
\end{verbatim}

\begin{verbatim}
## Confusion Matrix and Statistics
## 
##           Reference
## Prediction    0    1
##          0 2095    1
##          1  199    5
##                                           
##                Accuracy : 0.913           
##                  95% CI : (0.9008, 0.9242)
##     No Information Rate : 0.9974          
##     P-Value [Acc > NIR] : 1               
##                                           
##                   Kappa : 0.0428          
##                                           
##  Mcnemar's Test P-Value : <2e-16          
##                                           
##             Sensitivity : 0.833333        
##             Specificity : 0.913252        
##          Pos Pred Value : 0.024510        
##          Neg Pred Value : 0.999523        
##              Prevalence : 0.002609        
##          Detection Rate : 0.002174        
##    Detection Prevalence : 0.088696        
##       Balanced Accuracy : 0.873293        
##                                           
##        'Positive' Class : 1               
## 
\end{verbatim}

We see the best cutoff for the highest sensitivity is at 0.8. With this,
the performance of our model on the bbc\_test data set is 91\% accurate
overall, and our sensitivity is 83\% with a specificity of 91\%

\hypertarget{conclusion}{%
\section{Conclusion}\label{conclusion}}

\emph{Concludes with a summary of the aim and results. Discusses
alternative methods that can be used.}

Based on our analysis, the \textbf{ADD MODEL}

\textbf{Add details on if it is better to use model or just send
campaign to everyone in the list.}

\hypertarget{stuff-from-the-pdf}{%
\section{STUFF FROM THE PDF}\label{stuff-from-the-pdf}}

Summarize the results of your analysis for the three models. The
training, testing, and prediction data can be found on Blackboard.

Interpret the results of the models. In particular, for models the
influential covariates and their coefficients, provide insights.

BBBC is considering a similar mail campaign in the Midwest where it has
data for 50,000 customers. Such mailings typically promote several
books. The allocated cost of the mailing is \$0.65/addressee (including
postage) for the art book, and the book costs \$15 to purchase and mail.
The company allocates overhead to each book at 45\% of cost. The selling
price of the book is \$31.95. Based on the model, which customers should
Bookbinders target? How much more profit would you expect the company to
generate using these models as compare to sending the mail offer to the
entire list.

Please also summarize the advantages and disadvantages of the three
models, as you experienced in the modeling exercise. Should the company
develop expertise in either (or all) of these methods to develop
in-house capability to evaluate its direct mail campaigns.

How would you simplify and automate your recommended method(s) for
future modeling efforts at the company?

\end{document}
